\documentclass[24pt,pdf,hyperref={unicode},aspectratio=169]{beamer}
\usepackage[utf8]{inputenc}
\usepackage{etex}
\usepackage{aiml}
\usepackage{ctable}
\usepackage{xifthen}
\usepackage{xparse}

\newcommand{\ImageSizeAlone}{0.6cm}
\newcommand{\ImageSizeFuzzy}{0.4cm}

\newcommand{\fuzzyimage}[2]{
\ifthenelse{\isempty{#2}}
{
\ifmmode{\begin{array}{c}\includegraphics[width=\ImageSizeAlone]{HOMM5/#1.jpg}\end{array}}
\else{\includegraphics[width=\ImageSizeAlone]{HOMM5/#1.jpg}}
\fi
}
{
\dfrac{#2}{\includegraphics[width=\ImageSizeFuzzy]{HOMM5/#1.jpg}}
}
}

\newcommand{\ar}{}
\DeclareDocumentCommand\ar{O{}}{\fuzzyimage{archer}{#1}}
\newcommand{\fo}{}
\DeclareDocumentCommand\fo{O{}}{\fuzzyimage{footman}{#1}}
\newcommand{\mo}{}
\DeclareDocumentCommand\mo{O{}}{\fuzzyimage{monk}{#1}}
\newcommand{\sk}{}
\DeclareDocumentCommand\sk{O{}}{\fuzzyimage{skeleton}{#1}}
\newcommand{\gh}{}
\DeclareDocumentCommand\gh{O{}}{\fuzzyimage{ghost}{#1}}
\newcommand{\lc}{}
\DeclareDocumentCommand\lc{O{}}{\fuzzyimage{lich}{#1}}


\newcommand{\vc}{\centering\arraybackslash}

\newcommand{\legendy}{1.6}

\newcommand{\plotandlegend}[4]
{
\draw[#1, ultra thick] plot file {Plots/#2.txt};
\node[#1] at (9,\legendy-0.2*#3) {$
\begin{array}{p{1.5cm}p{1cm}} 
$#4$ & \input{Plots/#2.value.txt}\\
\end{array}
$};
}

\newcommand{\batteryplot}[1]
{
\plotandlegend{red}{4#1}{0}{4}
\plotandlegend{blue}{2_plus_2#1}{1}{2+2}
\plotandlegend{green}{2_mult_2#1}{2}{2\cdot 2}
}

\begin{document}


\section{Нечеткий вывод}


\begin{frame}\frametitle{Нечеткая импликация}

\uncover<+->{$$
\begin{array}{c c|c}
x & y & x\rightarrow y \\
\hline
0 & 0 & 1 \\
0 & 1 & 1 \\
1 & 0 & 0 \\
1 & 1 & 1 \\
\end{array}
$$}

\uncover<+->{$$
x \imp{}{KD} y=\neg x \fvee y=\max(1-x,y)
$$}

\uncover<+->{$$
x \imp{}{G} y = \left\{\begin{array}{l l}
                        \min(1,y/x), & x>0 \\
                        1 & x=0
                       \end{array}\right.
$$}
\end{frame}




\begin{frame}[t]\frametitle{Нечеткий вывод}

\only<1-3>{$$
\left(\sk[0.9]+\gh[0.4]+\lc[0.1]\right)
$$}
\only<3->{$$
\left(\sk[0.8]+\gh[0.5]+\lc[0.2]\right)\rightarrow\left(\ar[0.7]+\fo[0.4]+\mo[0.1]\right)
$$}
\only<2,3>{$$
\begin{array}{ >\vc m{0.7cm}| >\vc m{0.7cm} >\vc m{0.7cm} >\vc m{0.7cm}}
$\rho$  & \ar & \fo  & \mo \\
\hline
\vskip 1pt
\sk & ? & ? & ? \\
\gh & ? & ? & ? \\
\lc & ? & ? & ? \\
\end{array}
$$}
\only<4-11>{$$
\begin{array}{ >\vc m{0.8cm}| >\vc m{0.7cm} >\vc m{0.7cm} >\vc m{0.7cm}}
$\imp{N,H}{KD}$  & \ar   & \fo & \mo \\
\hline
\vskip 1pt
\sk & \uncover<8->{0.7} & \uncover<9->{0.4} & \uncover<10->{0.2} \\
\gh & \uncover<11>{0.7 & 0.5 & 0.5} \\
\lc & \uncover<11>{0.8 & 0.8 & 0.8} \\
\end{array}
$$}
\only<5-11>{$$N\rightarrow H \Rightarrow
\mu_{\imp{N,H}{KD}}(n,h)=\imp{}{KD}\left[\mu_N(n),\mu_H(h)\right]=\max\left[1-\mu_N(n),\mu_H(h)\right]
$$}
\only<6-7>{$\mu_{\imp{N,H}{KD}}\left(\sk,\ar\right)$}
\only<7>{$=\imp{}{KD}(0.8,0.7)=\max(1-0.8,0.7)=0.7$}

\only<12->{$$
\begin{array}{ >\vc m{0.8cm}| >\vc m{0.7cm} >\vc m{0.7cm} >\vc m{0.7cm}}
$\imp{N,H}{G}$  & \ar   & \fo & \mo \\
\hline
\vskip 1pt
\sk & \uncover<13>{0.875 & 0.5 & 0.125 } \\
\gh & \uncover<13>{ 1 & 0.8 & 0.2} \\
\lc & \uncover<13>{ 1 &  1 & 0.5} \\
\end{array}
$$}
\only<12->{$$N\rightarrow H \Rightarrow
\mu_{\imp{N,H}{G}}(n,h)=\imp{}{G}[\mu_N(n),\mu_H(h)]=\min\left[1,y/x\right]
$$}
\end{frame}

\begin{frame}[t]\frametitle{Нечеткий вывод}
\begin{center}$\left(\sk[0.8]+\gh[0.5]+\lc[0.2]\right)\rightarrow\left(\ar[0.7]+\fo[0.4]+\mo[0.1]\right)$\end{center}
\begin{columns}[t]

\renewcommand{\ImageSizeAlone}{0.4cm}

\column{0.4\textwidth}
$$\begin{array}{ >\vc m{0.8cm}| >\vc m{0.7cm} >\vc m{0.7cm} >\vc m{0.7cm}}
$\imp{N,H}{KD}$  & \ar   & \fo & \mo \\
\hline
\vskip 1pt
\sk & 0.7 & 0.4 & 0.2 \\
\gh & 0.7 & 0.5 & 0.5 \\
\lc & 0.8 & 0.8 & 0.8 \\
\end{array}
$$
$$
\begin{array}{ >\vc m{0.8cm}| >\vc m{0.7cm} >\vc m{0.7cm} >\vc m{0.7cm}}
$\imp{N,H}{G}$  & \ar   & \fo & \mo \\
\hline
\vskip 1pt
\sk & 0.875 & 0.5 & 0.125 \\
\gh &  1 & 0.8 & 0.2 \\
\lc &  1 &  1 & 0.5 \\
\end{array}
$$

\column{0.6\textwidth}

\renewcommand{\ImageSizeAlone}{0.6cm}

\only<2->{$$
\mu_B(b)=\max_{a\in\mathbb{A}}\left[\mu_A(a) \mu_\rho(a,b)\right]
$$}
\only<2-3>{$$
\imp{N,H}{KD}\left(\sk[0.8]+\gh[0.5]+\lc[0.2]\right)=
$$
$$
=\left( \ar[0.56]+\fo[0.32]+\mo[0.25] \right)
$$}
\only<3>{$$
\imp{N,H}{G}(\ldots)=\left( \ar[0.7]+\fo[0.4]+\mo[0.1] \right)
$$}
\only<4>{$$
\imp{H,N}{KD}\left(\sk[0.9]+\gh[0.4]+\lc[0.2]\right)=
$$
$$
=\left( \ar[0.63]+\fo[0.36]+\mo[0.18] \right)
$$
$$
\imp{N,H}{G}(\ldots)=\left( \ar[0.72]+\fo[0.45]+\mo[0.1125] \right)
$$}
\only<5>{$$
\imp{N,H}{KD}\left(\sk[0.1]+\gh[0.2]+\lc[0.8]\right)=
$$
$$
=\left( \ar[0.64]+\fo[0.64]+\mo[0.64] \right)
$$
$$
\imp{N,H}{G}(\ldots)=\left( \ar[0.8]+\fo[0.8]+\mo[0.4] \right)
$$}
\end{columns}
\end{frame}

\section{Пояснения к нечеткому выводу}

\begin{frame}\frametitle{Множества, отношения и предикаты}

$$
\begin{array}{c p{1cm} c}
\uncover<+->{\mathbb{A},\ A\subset \mathbb{A},\ a\in A}
&&
\uncover<+->{(\mathbb{A},P_A)} \\
&& 
\uncover<+->{P_A:\mathbb{A}\rightarrow\{0,1\}} \\
&& 
\uncover<+->{P_A(a)\Leftrightarrow a\in A}\\[0.5cm]
\uncover<+->{\mathbb{M},\ M\fsubset \mathbb{M},\ m\fin M}
&&
\uncover<+->{(\mathbb{M},\mu_M)} \\
&&
\uncover<+->{\mu_M:\mathbb{M}\rightarrow[0,1]}\\
&&
\uncover<+->{\mu_M(m)=m\fin M} \\[0.5cm]
\uncover<+->{\rho\subset\mathbb{A}\times\mathbb{B}}
&&
\uncover<+->{(\mathbb{A}\times\mathbb{B},P_\rho)} \\
&&
\uncover<+->{P_\rho:\mathbb{A}\times\mathbb{B}\rightarrow\{0,1\}}\\
&&
\uncover<+->{P_\rho(a,b)\Leftrightarrow (a,b)\in\rho} \\[0.5cm]
\uncover<+->{\sigma\fsubset\mathbb{M}\times\mathbb{N}}
&&
\uncover<+->{(\mathbb{M}\times\mathbb{N},\mu_\sigma)} \\
&&
\uncover<+->{\mu_\sigma:\mathbb{M}\times\mathbb{N}\rightarrow[0,1]}\\
&&
\uncover<+->{\mu_\sigma(a,b)=(a,b)\fin\sigma} \\[0.5cm]


\end{array}
$$
\end{frame}

\begin{frame}\frametitle{Вывод формулы нечеткого вывода}
$$
\uncover<1->{
\begin{array}{ >\vc m{0.8cm}| >\vc m{0.7cm} >\vc m{0.7cm} >\vc m{0.7cm}}
$Ef$  & \ar   & \fo & \mo \\
\hline
\vskip 1pt
\sk & 1 & 1  & 0 \\
\gh &  1 & 1 & 0 \\
\lc &  0 &  0 & 0 \\
\end{array}
}
\ \ \ 
\uncover<4->{
\begin{array}{ >\vc m{0.8cm}| >\vc m{0.7cm} >\vc m{0.7cm} >\vc m{0.7cm}}
$Ex$  & \ar   & \fo & \mo \\
\hline
\vskip 1pt
\sk & 1 & 1  & 0 \\
\gh &  1 & 1 & 0 \\
\lc &  1 &  1 & 1 \\
\end{array}
}
$$
\uncover<2->{$$
\begin{array}{ >\vc m{0.8cm}| >\vc m{0.7cm} >\vc m{0.7cm} >\vc m{0.7cm}}
  & \sk   & \gh & \lc \\
\hline
\vskip 1pt
$N(n)$  &  1 & 1 & 0 \\
\end{array}
\ \ \ 
\begin{array}{ >\vc m{0.8cm}| >\vc m{0.7cm} >\vc m{0.7cm} >\vc m{0.7cm}}
 & \ar   & \fo & \mo \\
\hline
\vskip 1pt
$H(h)$  &  1 & 1 & 0 \\
\end{array}
$$}
{\small
\uncover<3->{$$
Ex(n,h):=N(n)\rightarrow H(h) \approx Ef(n,h)
$$}
\uncover<5->{$$
Ex(n,h)=(n\in N)\rightarrow (h\in H)
$$}
\uncover<6->{$$
\foperation{Ex(n,h)}=(n\fin N) \imp{}{KD} (h\fin H) = \imp{}{KD}\left[\mu_N(n),\mu_H(h)\right]
$$}}
 
\end{frame}



\section{Нечеткие числа}

\begin{frame}\frametitle{Нечеткие числа}
$A\fsubset\mathbb{R}$ -- нечеткое число, если:
\begin{itemize}[<+->]
\item $A$ -- нормировано, т.е. 
$$
\exists!\ceil{A}\in\mathbb{R}\ \mu_A(\ceil{A})=1;
$$
\item $A$ -- выпукло, т.е. 
$$
\forall\ x,y\ \mu_A(\lambda x+(1-\lambda)y)\ge\min\left(\mu_A(x),\mu_a(y)\right);
$$
\end{itemize}
\uncover<+->{
\begin{tikzpicture}[x=2cm,y=2cm]
\draw[->,thick] (0,0.01) -- (4.5,0.01) node[below] {$x$};
\draw[->,thick] (0,0) -- (0,1.2) node[left] {$\mu$};
\node at(2,1.1) {$A\approx 2$};
\draw[red,thick] plot file {Plots/2.txt};
\only<+>{
\draw[dotted] (2,1) -- (0,1) node[left] {$1$};
\draw[dotted] (2,1) -- (2,0) node[below] {$2$};
}
\only<+>{
\draw[fill=black] (1.5,0.778) circle(2pt);
\draw[fill=black] (2.5,0.778) circle(2pt);
\draw (1.5,0.778) -- (2.5,0.778);
}
\only<+->{
\draw[fill=black] (1.5,0.778) circle(2pt);
\draw[fill=black] (0.5,0.105) circle(2pt);
\draw (0.5,0.105)--(1.5,0.778);
}
\only<+->{
\draw[dotted] (1.5,0.778) -- (1.5,0.105) -- (0.5,0.105);
 }
\end{tikzpicture}
}
\end{frame}



\section{Принцип расширения}

\begin{frame}\frametitle{Принцип расширения}
\uncover<+->{$$
A\fsubset\mathbb{A},\ B\fsubset\mathbb{B},\ C\fsubset\mathbb{C}
$$}
\uncover<+->{$$
\star:\mathbb{A}\times\mathbb{B}\rightarrow\mathbb{C}
$$
$$
C=A\star B
$$}
\uncover<+->{$$
\rho_\star\subset(\mathbb{A}\times\mathbb{B})\times\mathbb{C}
$$
$$
C=\rho_\star(A,B)
$$}
\uncover<+->{$$
\mu_C(c)=\underset{(a,b)\in\mathbb{A\times B}}{S}\left[T\left(\mu_{(A,B)}(a,b),\mu_{\rho_\star}((a,b),c)\right)\right]
$$}
\uncover<+->{$$
\mu_C(c)=\underset{(a,b)\in\mathbb{A\times B}}{S}\left[(a\fin A)\fwedge(b\fin B)\fwedge((a,b,c)\fin\rho_\star)\right]
$$}
\uncover<+->{$$
\mu_C(c)=\max_{a,b\ :\ a\star b=c}\left[\mu_A(a)\mu_B(b)\right]
$$}
\end{frame}



\begin{frame}[t]\frametitle{Принцип расширения}
\begin{columns}[t]
\column{0.4\textwidth}
\uncover<1->{
$$
\mathbb{Z}_4=\{0,1,2,3\}
$$
}
\uncover<2->{
$$
\begin{array}{l|l l l l}
+ & 0 & 1 & 2 & 3 \\
\hline
0 & \alert<7>{0} & \alert<9>{1} & \alert<11>{2} & \alert<13>{3} \\
1 & \alert<9>{1} & \alert<11>{2} & \alert<13>{3} & \alert<7>{0} \\
2 & \alert<11>{2} & \alert<13>{3} & \alert<7>{0} & \alert<9>{1} \\
3 & \alert<13>{3} & \alert<7>{0} & \alert<9>{1} & \alert<11>{2} \\
\end{array}
$$
}

\uncover<3->{
$$
\begin{array}{l|l l l l}
\cdot & 0 & 1 & 2 & 3 \\
\hline
0 & \alert<18>{0} & \alert<18>{0} & \alert<18>{0} & \alert<18>{0} \\
1 & \alert<18>{0} & \alert<19>{1} & \alert<20>{2} & \alert<21>{3} \\
2 & \alert<18>{0} & \alert<20>{2} & \alert<18>{0} & \alert<20>{2} \\
3 & \alert<18>{0} & \alert<21>{3} & \alert<20>{2} & \alert<19>{1} \\
\end{array}
$$
}

\column{0.6\textwidth}
\only<4-15>{$$
A=\frac{0.5}{0}+\frac{0.9}{1}+\frac{0.1}{2}+\frac{0.1}{3}
$$
$$
B=\frac{0.1}{0}+\frac{0.1}{1}+\frac{0.9}{2}+\frac{0.5}{3}
$$
$$
C=A+B
$$}
\only<5-15>{$$
\mu_C(c)=\max_{a,b\ :\ a+b=c}\left[\mu_A(a)\mu_B(b)\right]
$$}
\only<8-14>{$$
C=\frac{0.45}{0}+\uncover<10->{\frac{0.09}{1}+}\uncover<12->{\frac{0.45}{2}+}\uncover<14->{\frac{0.81}{3}}
$$}
\only<15>{$$
C=\frac{0.09}{1}+\frac{0.45}{2}+\frac{0.81}{3}+\frac{0.45}{0}
$$}
\only<6-7>{$$
\mu_C(0)=\uncover<7->{
\max\left(\begin{array}{l l}
0.5\cdot 0.1, & 0.9\cdot0.5,\\
0.1\cdot0.9, &0.1\cdot0.1
\end{array}\right)=}
$$}
\only<7>{$$
=\max(0.05,\ 0.45,\ 0.09,\ 0.01)=0.45
$$}
\only<9>{$$
\mu_C(1)=\max(0.05,0.09,0.05,0.09)
$$}
\only<11>{$$
\mu_C(2)=\max(0.45,0.09,0.01,0.05)
$$}
\only<13>{$$
\mu_C(3)=\max(0.25,0.81,0.01,0.01)
$$}
\only<16->{$$
A=\frac{0.1}{0}+\frac{0.5}{1}+\frac{0.9}{2}+\frac{0.1}{3}
$$
$$
C=A\cdot A
$$
$$
\mu_C(c)=\max_{a,b\ :\ a\cdot b=c}\left[\mu_A(a)\mu_B(b)\right]
$$}
\only<17-22>{$$
C=\frac{\uncover<18->{0.81}}{0}+\frac{\uncover<19->{0.25}}{1}+\frac{\uncover<20->{0.45}}{2}+\frac{\uncover<21->{0.05}}{3}
$$}

\end{columns}


\end{frame}

\section{Нечеткая арифметика}


\begin{frame}\frametitle{Теорема Дюбуа-Прайда}

Пусть $A,B$ -- нечеткие числа, и операции сложения, умножения, вычитания и деления определены по принципу расширения. Тогда:

\begin{enumerate}[<+->]
\item $A+B$, $A-B$, $A\cdot B$, $A/B$ -- нечеткие числа;
\item Выполняется

$$
\begin{array}{l}
\ceil{A+B}=\ceil{A}+\ceil{B}\\
\ceil{A-B}=\ceil{A}-\ceil{B}\\
\ceil{A\cdot B}=\ceil{A}\cdot\ceil{B}\\
\ceil{A/B}=\ceil{A}/\ceil{B}.
\end{array}
$$
\end{enumerate}
\uncover<+->{$$
|A|=\int_{a\in\mathbb{R}}\mu_A(a)ada
$$}
\end{frame}

\begin{frame}\frametitle{Нечеткие операции}
\begin{tikzpicture}[x=1cm,y=3cm]

\draw[->,thick] (0,0.01) -- (10.5,0.01) node[below] {$x$};
\draw[->,thick] (0,0) -- (0,1.2) node[left] {$\mu$};

\only<1-2>{
\plotandlegend{orange}{2}{-1}{2}
\draw[dotted] (2,1)--(2,0) node[below] {2};
\draw[dotted] (2,1)--(0,1) node[left] {1};
}
\only<2-5>{
\plotandlegend{red}{4}{0}{4}
\draw[dotted] (4,1)--(4,0) node[below] {4};
\draw[dotted] (4,1)--(0,1) node[left] {1};
}

\only<3-5>{
\plotandlegend{blue}{2_plus_2}{1}{2+2}
}
\only<4-5>{
\plotandlegend{green}{2_mult_2}{2}{2\cdot 2}
}
\only<5>{
\plotandlegend{purple}{8_div_2}{3}{8/2}
}

\only<6->{
\node at (2,1.7) {$
\begin{array}{l c l}
\mu_A(x) &=& \only<6-10,12->{exp\left[\left(\frac{x-\ceil{A}}{K}\right)^2\right]}\only<11>{\max\left[0,1-\left(\frac{x-\ceil{A}}{K}\right)^2\right]}\\
K &=& \only<6,8,10->{1} \only<7>{0.3} \only<9>{3} \\
T(x,y)&=& \only<6-12,14->{xy} \only<13>{\min(x,y)} \\
S(x,y)&=& \only<6-14>{\max(x,y)} \only<15>{x+y-xy} \\
\end{array}
$};
}

\only<6,8,10,12,14>{
\batteryplot{}
}

\only<7>{
\batteryplot{_K03}
}

\only<9>{
\batteryplot{_K3}
}

\only<11>{
\batteryplot{_Q}
}

\only<13>{
\batteryplot{_min}
}

\only<15>{
\batteryplot{_nomax}
}
\end{tikzpicture}
\end{frame}




\renewcommand{\legendy}{2}

\begin{frame}\frametitle{Размывание}
\begin{tikzpicture}[x=0.9cm,y=3cm]
\draw[->,thick] (0,0.01) -- (12.1,0.01) node[below] {$x$};
\draw[->,thick] (0,0) -- (0,1.2) node[left] {$\mu$};


\plotandlegend{black}{6}{-1}{6}


\only<2->{
\plotandlegend{blue}{3_times_2}{0}{3+3}
}
\only<3->{
\plotandlegend{purple}{2_times_3}{1}{2+2+2}
}
\only<4->{
\plotandlegend{orange}{1_times_6}{2}{1+\ldots+1}
}
\only<5->{
\plotandlegend{teal}{2_mult_3}{3}{2\cdot 3}
}
\only<6>{
\plotandlegend{green}{6_mult_1}{4}{6 \cdot 1}
}

\end{tikzpicture}
\end{frame}

\begin{frame}\frametitle{Размывание}

$$
C=\frac{A+B}{B}
$$
$$
f(a,b)=\frac{a+b}{b}
$$
$$
\mu_C(c)=\max_{a,b\ f(a,b)=c}\mu_A(a)\mu_B(b)
$$

\begin{tikzpicture}[x=0.9cm,y=3cm]
\draw[->,thick] (0,0.01) -- (12.1,0.01) node[below] {$x$};
\draw[->,thick] (0,0) -- (0,1.2) node[left] {$\mu$};
\draw[red, ultra thick] plot file {Plots/F_direct.txt};
\draw[green, ultra thick] plot file {Plots/F_ext.txt};
\end{tikzpicture}

\end{frame}



\end{document}
