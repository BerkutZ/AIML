\documentclass[24pt,pdf,hyperref={unicode},aspectratio=169]{beamer}
\usepackage[utf8]{inputenc}
\usepackage{aiml}
\usepackage{framed}


\newcommand{\ce}{$Ц$}
\newcommand{\za}{$З$}
\newcommand{\av}{$A$}
\newcommand{\na}{$H$}
\newcommand{\uga}{$У.1$}
\newcommand{\ugb}{$У.4$}
\newcommand{\ra}{$Р$}
\newcommand{\kr}{$K$}

\newcommand{\law}[1]{
\begin{framed}
{ 

#1
}
\end{framed}
}

\begin{document}



\section{Экспертные системы}


\begin{frame}[t]\frametitle{Составление правил}
\begin{columns}
\column[T]{0.6\textwidth}
\only<1-6>{\law{
\footnotesize
{\bf Статья 166. Неправомерное завладение автомобилем или иным транспортным средством без цели хищения}

\begin{enumerate}
\item Неправомерное \alert<2-4>{завладение автомобилем} или иным транспортным средством \alert<2-4>{без цели хищения} (угон) -- наказывается...
\item[2-3.] ...
\item[4.] Деяния, предусмотренные частями первой, второй или третьей настоящей статьи, совершенные \alert<5-6>{с применением насилия}, опасного для жизни или здоровья, либо с угрозой применения такого насилия, --- ...
\end{enumerate}
}}

\only<7-8>{\law{
{\bf Статья 162. Разбой}
\begin{enumerate}
\item Разбой, то есть нападение в целях хищения чужого имущества, совершенное с применением насилия, опасного для жизни или здоровья, либо с угрозой применения такого насилия, --- ...
\end{enumerate}
}}

\only<9-10>{\law{
{\bf Статья 158. Кража}
\begin{enumerate}
\item Кража, то есть тайное хищение чужого \newline имущества, --- ...
\end{enumerate}
}}
\column[T]{0.4\textwidth}

\vspace{1cm}

\only<3>{$\za\wedge\av\wedge\neg\ce\rightarrow \uga\ $ }

\only<4-> {$\neg\za\vee\neg\av\vee\ce\vee \uga\ $ }

\only<6-> {$\neg\za\vee\neg\av\vee\ce\vee\neg \na \vee \ugb\ $ }

\only<8->{$\neg\na\vee\neg\ce\vee \ra\ $ }

\only<10->{$\neg\za\vee\neg\ce\vee\na\vee\kr\ $}
\end{columns}


\end{frame}

\begin{frame}\frametitle{Логический вывод}
\begin{tikzpicture}[x=5cm,y=3cm]
\uncover<1->{
\node(s) at(0,0) {$\begin{array}{l}
\neg\za\vee\neg\av\vee\ce\vee \uga \\
\neg\za\vee\neg\av\vee\ce\vee\neg \na \vee \ugb \\
\neg\na\vee\neg\ce\vee \ra \\
\neg\za\vee\neg\ce\vee\na\vee\kr 
\end{array}$};
}


\uncover<3->{
\node(z) at (1,0) {$\begin{array}{l}
\neg\av\vee\ce\vee \uga \\
\neg\av\vee\ce\vee\neg \na \vee \ugb \\
\neg\na\vee\neg\ce\vee \ra \\
\neg\ce\vee\na\vee\kr 
\end{array}$};
}

\uncover<2->{
\path (s) edge[->] node[above]{$\za$} (z);
}




\uncover<5->{
\node(z-n) at (0,-1) {$\begin{array}{l}
\neg\av\vee\ce\vee \uga \\
\neg\ce\vee\kr 
\end{array}$};
}

\uncover<4->{
\path (z) edge[->] node[above]{$\neg\na\ $} (z-n);
}




\uncover<7->{
\node(z-n-c) at (0,-1.5) {$\begin{array}{l}
\neg\av\vee \uga \\
\end{array}$};
}

\uncover<6->{
\path (z-n) edge[->] node[right]{$\neg\ce\ $} (z-n-c);
}

\uncover<9->{
\node(z-n-c-a) at (0,-2) {\large ?};
}

\uncover<8->{
\path (z-n-c) edge[->] node[right]{$\neg\av\ $} (z-n-c-a);
}

\uncover<11->{
\node(z+n) at (1,-1) {$\begin{array}{l}
\neg\av\vee\ce\vee\ugb \\
\neg\ce\vee \ra \\
\end{array}$};
}

\uncover<10->{
\path (z) edge[->] node[right]{$\na$} (z+n);
}

\uncover<13->{
\node(z+n+a) at (1,-2) {$\begin{array}{l}
\ce\vee\ugb \\
\neg\ce\vee \ra \\
\end{array}$};
}

\uncover<12->{
\path (z+n) edge[->] node[right]{$\av $} (z+n+a);
}

\end{tikzpicture}
\end{frame}

\begin{frame}\frametitle{Работа экспертной системы}
\begin{itemize}
\item<+->[M:] ?
\item<+->[O:] $\za!$
\item<+->[М:] $\na?$
\item<+->[O:] {\it why?}
\item<+->[М:] ...
\item<+->[М:] $\na?$
\item<+->[O:] $\neg\na!$
\item<+->[М:] $\ce?$
\item<+->[O:] $\ce!$
\item<+->[М:] $\kr!$
\item<+->[O:] {\it how?}
\item<+->[М:] ...
\end{itemize}
\end{frame}
\end{document}
