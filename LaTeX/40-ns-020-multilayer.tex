\documentclass[24pt,pdf,hyperref={unicode},aspectratio=169]{beamer}
\usepackage[utf8]{inputenc}
\usepackage[russian]{babel}
\usepackage{graphics}
\usepackage{amssymb}
\usepackage{xstring}
\usepackage{multirow}
\usepackage{tikz}
\usepackage[all]{xy}


\newcommand{\dd}[2]{\frac{\partial #1}{\partial #2}}
\begin{document}
\tikzstyle{neu}=[circle,fill=blue!50,minimum size=0.8cm]



\section{Многослойный персептрон}

\begin{frame}\frametitle{Функция XOR}
\begin{columns}
\column{0.5\textwidth}
$$
\begin{array}{c c | c}
x_1 & x_2 & x_1\oplus x_2 \\
\hline
-1 & -1 & -1 \\
-1 & 1 & 1 \\
1 & -1 & 1 \\
1 & 1 & -1 \\
\end{array}
$$\\[1cm]
$$
\left\{
\begin{array}{l l l l l l}
 w_0 - w_1 - w_2  & > & 0 \\
 w_0 + w_1 - w_2 & < & 0 \\
 w_0 - w_1 + w_2 & < & 0 \\
 w_0 + w_1 + w_2 & > & 0 \\
\end{array}
\right.
$$
\column{0.5\textwidth}
\begin{tikzpicture}[y=1.5cm, x=1.5cm]
\draw[->] (-1.2,0) -- (1.4,0) node[below]{$x_1$};
\draw[->] (0,-1.2) -- (0,1.4) node[left]{$x_2$};
\draw[fill=black] (-1,-1) circle(3pt);
\draw[fill=white] (-1,1) circle(3pt);
\draw[fill=white] (1,-1) circle(3pt);
\draw[fill=black] (1,1) circle(3pt);
\end{tikzpicture}
\end{columns}
\end{frame}

\begin{frame}\frametitle{Функция XOR}

$$
x_1\oplus x_2=(x_1\wedge \neg x_2)\vee (x_2\wedge \neg x_1)
$$\\[1cm]
\uncover<+->{}
\begin{tikzpicture}
\node (x-0) at (0,4) {$1$};
\node (x-1) at (0,2) {$x_1$};
\node (x-2) at (0,0) {$x_2$};

\node[neu] (n-1) at (3,3) {};
\node[neu] (n-2) at (3,1) {};

\node(y-0) at (5,5) {$1$};
\node(y-1) at (5,3) {$x_1\wedge \neg x_2$};
\node(y-2) at (5,1) {$x_2\wedge \neg x_1$};

\node[neu] (n-3) at (8,3) {};
\node(out) at (10,3) {$x_1\oplus x_2$};

\path (x-0) edge[->] node[above,near end] {-1} (n-1);
\path (x-1) edge[->] node[above,near end] {1} (n-1);
\path (x-2) edge[->] node[above,near end] {-1} (n-1);
\path (n-1) edge[->] (y-1);

\path (x-0) edge[->] node[below,near end] {-1} (n-2);
\path (x-1) edge[->] node[below,near end] {-1} (n-2);
\path (x-2) edge[->] node[below,near end] {1}  (n-2);
\path (n-2) edge[->] (y-2);

\path (y-0) edge[->] node[above,midway] {1} (n-3);
\path (y-1) edge[->] node[above,midway] {1}  (n-3);
\path (y-2) edge[->] node[above,midway] {1}  (n-3);
\path (n-3) edge[->] (out);
\end{tikzpicture} 
\end{frame}

\begin{frame}\frametitle{Многослойный персептрон}
\begin{tikzpicture}[shorten >=1pt,->,draw=black!50, node distance=\layersep]
     \tikzstyle{every pin edge}=[<-,shorten <=1pt]
     \tikzstyle{neuron}=[circle,fill=black!25,minimum size=17pt,inner sep=0pt]
     \tikzstyle{input neuron}=[neuron, fill=green!50];
     \tikzstyle{output neuron}=[neuron, fill=red!50];
     \tikzstyle{hidden neuron}=[neuron, fill=blue!50];
     \tikzstyle{annot} = [text width=4em, text centered]
 
 	
 	\foreach \num in {1,...,5}
 		\node (x-\num) at (0,5-\num) {$x_{\num}$};
 		
 	\foreach \num in {1,...,4}
 	{
 		\node[neu] (n-\num) at (2,4.5-\num) {};
 		\node (y-\num) at (3,4.5-\num) {$y_{\num}$};
 		\path (n-\num) edge (y-\num);
 	}
 	
 	\foreach \num in {1,...,5}
 	{
	 	\node[neu] (m-\num) at (5,5-\num) {};
 	 	\node (z-\num) at (6,5-\num) {$z_{\num}$};
 	 	\path (m-\num) edge (z-\num);
 	}
 	
 	
 	\foreach \num in {1,...,3}
 	{
	 	\node[neu] (p-\num) at (8,4-\num) {};
 	 	\node (o-\num) at (9,4-\num) {$o_{\num}$};
 	 	\path (p-\num) edge (o-\num);
 	} 	
 	
 	\foreach \st in {1,...,5} \foreach \ds in {1,...,4} \path (x-\st) edge (n-\ds);
 	\foreach \st in {1,...,4} \foreach \ds in {1,...,5} \path (y-\st) edge (m-\ds);
 	\foreach \st in {1,...,5} \foreach \ds in {1,...,3} \path (z-\st) edge (p-\ds);
\end{tikzpicture}
\end{frame}


\begin{frame}\frametitle{Постановка задачи}
{\bf Дано:}
\begin{tabular}{p{6cm} p{7.5cm}}
 $\mathcal{X}=(X_1,\ldots,X_k)$ & входные вектора, $X_i\in\mathbb{R}^n$\\[0.1cm]
 $\mathcal{A}=(A_1,\ldots,A_k)$ & правильные выходные вектора, $A_i\in\mathbb{R}^m$\\[0.1cm]
 $(\mathcal{X},\mathcal{A})$ & обучающая выборка  \\[0.1cm]
 $W$ & вектор весов нейронной сети \\[0.1cm]
 $N(W,X)$ & функция, соответствующая нейронной сети \\[0.1cm]
 $Y=N(W,X)$ & ответ нейронной сети, $Y\in\mathbb{R}^m$ \\[0.1cm]
 $D(Y,A) =\sum_{j=1}^{m} (Y[j]-A[j])^2$ & функция ошибки\\[0.1cm]
 $D_i(Y)=D(Y,A_i) $ & функция ошибки на $i$-ом примере \\[0.1cm]
  $E_i(W)=D_i(N(W,X_i))$ & ошибка сети на $i$-ом примере\\[0.1cm]
  $E(W)=\sum_{i=1}^{k}E_i(W) $ & ошибка сети на всей обучающей выборке \\[0.1cm]      
 \end{tabular}
{\bf Найти:}

вектор $W$ такой, что $E(W) \rightarrow \min$ (обучение на всей выборке)

вектор $W$ такой, что $E_i(W) \rightarrow \min$ (обучение на одном примере)

\end{frame}




\begin{frame}\frametitle{Алгоритм градиентного спуска}
\begin{enumerate}
 \item Инициализировать $x_1$ случайным значением из $\mathbb{R}$
 \item $i:=1$
 \item $x_{i+1}=x_i-\varepsilon f'(x_i)$
 \item $i++$
 \item ${\rm if}\ \ f(x_{i})-f(x_{i+1}) >c\ \ {\rm goto}\ 3$
\end{enumerate}
\end{frame}

\begin{frame}\frametitle{Алгоритм градиентного спуска}
\begin{enumerate}
 \item Инициализировать $W_1$ случайным значением из $\mathbb{R}^n$
 \item $i:=1$
 \item $W_{i+1}=W_i-\varepsilon \nabla f(W_i)$
 \item $i++$
 \item ${\rm if}\ \ f(W_{i})-f(W_{i+1})>c\ \ {\rm goto}\ 3$
\end{enumerate}
\end{frame}


\begin{frame}\frametitle{Онлайн и оффлайн обучение}
\uncover<+->{
Ошибка сети на $i$-м примере: $E_i(W)=D_i(N(W,X_i))$

Ошибка сети на всей выборке: $E(W)=\sum_{i=1}^{m}E_i(W)$ 
}

\uncover<+->{$$
\dd{E}{w_{ij}}=\dd{\sum_{i=1}^{m}E_i(W)}{w_{ij}}=\sum_{i=1}^{m}\dd{E_i}{w_{ij}}
$$}
\uncover<+->{$$
\nabla E = \sum_{i=1}^{m} \nabla E_i
$$}
\end{frame}

\begin{frame}\frametitle{Ускорение обучения}

\begin{itemize}
\item<+-> Правильный выбор коэффициента
$$
W_{i+1}=W_i-\varepsilon \nabla f(W_i)
$$

(начать с $\varepsilon=1$)

\item<+-> Повторение алгоритма градиентного спуска на каждом примере (3-5 раз)

\item<+-> Правило момента

$$
\Delta{i} = W_{i}-W_{i-1}
$$
$$
W_{i+1} = W_i-\varepsilon \nabla f(W_i)+\alpha\Delta_i, \ \ \ 
\alpha\approx 0.1
$$
\item<+-> Стимуляция нейронов

$$
W_{i+1} = (1-\gamma)W_i-\varepsilon \nabla f(W_i)+\alpha\Delta_i,\ \ \ 
\gamma \approx 10^{-4}
$$

\item<+-> Адаптивный выбор $\varepsilon$?

\item<+-> Сопряженные градиенты, обучение на гессиане??
\end{itemize}


\end{frame}


\end{document}