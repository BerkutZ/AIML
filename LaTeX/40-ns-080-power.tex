\documentclass[24pt,pdf,hyperref={unicode}]{beamer}
\usepackage[utf8]{inputenc}
\usepackage{aiml}

\begin{document}

\begin{frame}
\uncover<+->{$$
n(x_1,\ldots,x_n)=f\left(\sum_{i=1}^n x_iw_i\right)
$$}
\uncover<+->{$$
n(x_1,x_2)=x_1x_2?
$$}
\only<+>{$$
\begin{tikzpicture}[x=3cm]
\node[neu] (n) at (0,0) {$f(x)=x$};
\node (x) at (-1,0) {$x_1$};
\path (x) edge[->] node[above]{$x_2$} (n);
\node (y) at (1,0) {$x_1x_2$};
\path (n) edge[->] (y);
\end{tikzpicture}
$$}
\uncover<+->{$$
x_1x_2=\exp(\ln x_1 + \ln x_2)
$$}
\uncover<+->{$$
\begin{tikzpicture}[x=1.5cm]
\node (x1) at (0,1) {$x_1$};
\node (x2) at (0,-1) {$x_2$};
\node[neu] (n1) at (1,1) {$\ln$};
\node[neu] (n2) at (1,-1) {$\ln$};
\path (x1) edge[->] node[above] {$1$} (n1);
\path (x2) edge[->] node[above] {$1$} (n2);
\node[neu] (n3) at (2,0) {$\exp$};
\path (n1) edge[->] node[above] {$1$} (n3);
\path (n2) edge[->] node[above] {$1$} (n3);
\node (y) at (3,0) {$x_1x_2$};
\path (n3) edge[->] (y);
\end{tikzpicture}
$$}

\end{frame}


\begin{frame}
\bio{Hilbert}{David Hilbert}{Mathematische Probleme (1900)}
\end{frame}

\begin{frame}
Тринадцатая проблема Гильберта:\\[1cm]

Можно ли решить общее уравнение седьмой степени с помощью функций, зависящих только от двух переменных?
\end{frame}

\begin{frame}
\begin{small}
\doublebio
{Arnold}{Владимир Игоревич Арнольд}
{Kolmogorov}{Андрей Николаевич Колмогоров}
{
$ $\\[0.2cm]

О представлении непрерывных функций трех переменных суперпозициями непрерывных функций двух переменных (1959)\\[0.2cm]

О представлении непрерывных функций нескольких переменных в виде суперпозиций непрерывных функций одной переменной и сложения (1957)}
\end{small}
\end{frame}

\begin{frame}

$$
f(x_1,\dots,x_n)=\sum_{q=0}^{2 n} \Phi_q \left(\sum_{p=1}^n\psi_{q,p}(x_p)\right)
$$
\end{frame}

\begin{frame}\frametitle{Рекуррентные нейронные сети}
\end{frame}

\begin{frame}
\doublebio
{Siegelmann}{Hava Siegelmann}
{Sontag}{Eduardo Sontag}
{
Computation Beyond the Turing Limit (1995)
}
\end{frame}

\begin{frame}\frametitle{Проблема останова}
\begin{itemize}
\item Не существует программы $P$, которая по тексту программы $X$ и ее входу $Y$ определяла, остановится ли $X$ на $Y$, или не остановится.
\item Не существует программы $P$, которая по тексту программы $X$ определяет, что она останавливается на любых входных данных.
\item Существует способ присвоить каждой программе уникальный целочисленный номер.
\item Пусть $M$ -- множество всех номеров, соответствующих программам, которые всегда останавливаются.
\item Тогда не существует программы, которая бы по числу $x$ проверяла, что $x\in M$
\end{itemize}
\end{frame}


\end{document}