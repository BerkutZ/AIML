\documentclass[24pt,pdf,hyperref={unicode},aspectratio=169]{beamer}
\usepackage[utf8]{inputenc}
\usepackage{aiml}
\usepackage{pgfplots}



\begin{document}

\begin{frame}\frametitle{Логика высших порядков}

\uncover<+->{
$P$ является биекцией из $A$ в $B$
$$
\forall x \forall y \forall z\ P(x,y)\wedge P(x,z)\rightarrow Eq(y,z)
$$
$$
\forall x \exists y\ A(x)\wedge B(y)\wedge P(x,y)
$$
$$
\forall y \exists x\ A(x)\wedge B(y)\wedge P(x,y)
$$
$$
\forall x \forall y\ \forall z P(x,z) \wedge P(y,z) \rightarrow Eq(x,y)
$$}

\uncover<+->{
Равномощность $A$ и $B$:

$$
\exists P \left[ \forall x \forall y \forall z\ P(x,y)\wedge P(x,z)\rightarrow Eq(y,z) \right]\wedge\ldots
$$}


\end{frame}

\begin{frame}\frametitle{Принцип доказательства от противного}

\begin{itemize}
\item<+-> Необходимо доказать, что $\exists x P(x)$
\item<+-> Предположим, что $\forall x \neg P(x)$
\item<+-> Придем к противоречию
\item<+-> Следовательно, $\exists x P(x)$
\end{itemize}
\uncover<+->{
\begin{center}
Но чему равен $x$?
\end{center}
}
\end{frame}

\begin{frame}\frametitle{Модальная логика}

\uncover<+->{
Модальные операторы:

\begin{itemize}
\item $KA$ -- $А$ известно
\item $\diamond A$ -- $A$ возможно 
\end{itemize}}

\uncover<+->{
\begin{tabular}{l l l}
A1 & Принцип объективности знания & $KA\rightarrow A$ \\
A2 & Дистрибутивность знания и конъюнкции & $K(A\wedge B)\rightarrow KA\wedge KB$ \\
A3 & Принцип познаваемости мира & $A\rightarrow \diamond K A$ \\
\end{tabular}
}

\begin{itemize}
\item<+-> Предположим, $A\wedge\neg KA$ 
\item<+-> По A3, $\diamond K (A\wedge\neg KA)$ 
\item<+-> По A2, $\diamond ( KA \wedge K(\neg KA) )$ 
\item<+-> По А1, $\diamond ( KA \wedge \neg KA) $
\item<+-> Противоречие. Все уже познано.
\end{itemize}
\end{frame}

\end{document}